\subsection{Lyapunov Function }

In this first half we concentrate on phase locking state and exhibit a Lyapunov function.
One can define the arithmetical\footnote{ This is an aritmetical mean, however as $N \to \infty$ the quantity $\Bar{\omega}$) is a nonrandom number and equals the mean $\int d \mu(\omega) \omega$ by the strong law of large numbers.} mean frequency to be: 
\begin{equation}
\bar{\omega}:=\frac{1}{N} \sum_{i=1}^N \omega_i
\label{eq: meanOmega}
\end{equation}
and intuitively think that it is the common frequency of the phase locked component. In fact, 
\begin{align*}
    \frac{d}{d t}\left(\frac{1}{N} \sum_{i=1}^N \theta_i(t)\right)&= \left(\frac{1}{N} \sum_{i=1}^N \frac{d}{d t} \theta_i(t)\right) \\
\intertext{from equation \ref{eq:kura}:} \\ 
&= \left( \frac{1}{N} \sum_i \omega_i \right) + \frac{K}{N^2} \sum_{i,j} \sin(\theta_j(t) - \theta_i(t))
\end{align*}
Since $i, j = 1, \dots , N$, in the calculations one encounters the terms $\theta_i - \theta_j$ as well $\theta_j - \theta_i$ and since the sum is and odd function: 
\begin{equation*}
    \sin(\theta_i - \theta_j) = - \sin(\theta_j - \theta_i) 
\end{equation*}
hence the term $i,j$ and $j,i$ cancels out.
More precisely, we can separate the initial sum: 
\begin{equation*}
    \sum_{i>j} \sin(\theta_j - \theta_i) +  \sum_{i<j} \sin(\theta_j - \theta_i) 
\end{equation*}
exchanging $i$ with $j$ in the second one we get $\sum_{j<i} \sin(\theta_i - \theta_j) = - \sum_{j<i} \sin(\theta_j - \theta_i) $ which is exacly equal to the first term in the sum. 
we then are left with: 
\begin{equation}
    \frac{d}{d t}\left(\frac{1}{N} \sum_{i=1}^N \theta_i(t)\right) = \frac{1}{N} \sum_{i=1}^N \omega_i = \bar{\omega}
\end{equation}

Now one can consider the new variables $\phi_i$ in the reference frame rotating with $\omega=\Bar{\omega}$: 
\begin{equation*}
    \phi_i(t) = \theta_i(t) - \Bar{omega}t
\end{equation*}

so that the Kuramoto model becomes:
\begin{equation}
  \frac{d \phi_i}{d t}(t)=\left(\omega_i-\bar{\omega}\right)+\frac{K}{N} \sum_{j=1}^N \sin \left(\phi_j(t)-\phi_i(t)\right)  
  \label{eq:kura_ref}
\end{equation}

At this point we can introduce the Lyapunov function to study the nature of stable points (if they exists):

\begin{equation}
\mathcal{H}:=-\frac{K}{2 N} \sum_{i, j} \cos \left(\phi_i-\phi_j\right)-\sum_{i=1}^N\left(\omega_i-\bar{\omega}\right) \phi_i.
\label{eq:lyapunov}
\end{equation}
To be a good guess for a Lyapunov Function must satisfy:
\begin{equation}
\dot{\mathcal{H}}=\sum_{l=1}^N \frac{\partial \mathcal{H}}{\partial \phi_l} \frac{d \phi_l}{d t} \leq 0, 
\label{eq:lyap_cond}
\end{equation}
for any solution $\phi_i$. 


The term $\frac{d \phi_l}{d t}$ is given by eq. (\ref{eq:kura_ref}), while the term $\frac{\partial \mathcal{H}}{\partial \phi_l}$ is: 
\begin{align*}
    \frac{\partial \mathcal{H}}{\partial \phi_l} &= +\frac{K}{2 N} \sum_{i, j} \sin \left(\phi_i-\phi_j\right) (\delta_{i,l} - \delta_{j,l}\\
    &-\sum_{i=1}^N\left(\omega_i-\bar{\omega}\right) \delta_{i,l} \\
    &= -\frac{K}{2 N} \sum_{j} \sin \left(\phi_j-\phi_l\right)-\left(\omega_l-\bar{\omega}\right)\\
\end{align*}
one can notice that this term is equal to eq. \ref{eq:lyap_cond} except for a global minus sign\footnote{This means that $\phi_i = -\nabla \mathcal{H}$ we can represent the dynamics of the system thought the Lyapunov function which now can be considered as a potential }. 
Therefore:
\begin{equation*}
    \dot{\mathcal{H}}=-\sum_{l=1}^N \left(\frac{\partial \mathcal{H}}{\partial \phi_l}\right)^2  
\end{equation*}
which is non positive. 

Moreover we can write down the equation for the extreme points, and check their nature:
\begin{align*}
    \nabla \mathcal{H} = 0 = \frac{K}{2 N} \sum_{j} \sin \left(\phi_j-\phi_l\right)+\left(\omega_l-\bar{\omega}\right)\\
\end{align*}


